\documentclass[a4paper,12pt]{article}

% \usepackage{graphicx} 
% \usepackage{subfigure}
\begin{document}

\title{ORD Flight Delays Methodology}
\author{Aden Ramirez}
\maketitle

\pagebreak

\section{Research Question 1}
Can flight delays be predicted?

For this question we want a good overall look at stages of a flight, taxi, departure, landing, taxi, and arrival. The goal is to categorize if a flight will be delayed,
this is useful for airlines and people for expectations and planning around it. However there should also be an attempt at predicting the length of the delay to make 
the information more meaningful. This question will be solved with three different methods.

\subsection{Method 1}

The first method for this question that will be applied is logisitic regression. This method will have two possible categorical outputs, being the flight is delayed
or the flight is not delayed (Using $>$15m FAA definition). Logisitic regression is great for two class problems, and will fit well here.

\subsection{Method 2}
The second method for this question that will be applied is a random forest decision tree. This method will build an output of either delayed or not delayed.

% \subsection{Method 3}
% The third method for this research question is also a random forest decision tree, but using more categorical outputs. These being our delay groups that are used in the dataset. They are not delayed, 15 minutes - 30 minutes, and builds all 15 minute increments
% up to 180 minuts for a total of 12 possible groups. The power of the decision tress will allow to go beyond saying a flight is delayed or not, and offer a range following
% the ranges the DOT provides.

\section{Research Question 2}
What role do airports play in flight delays?

For this question focus is put on the airports themselves, focusing on only one airport we get a different scope looking at how arrivals from different ariports vary
or if they are delayed even when they depart early. This question will use much of the same methods, but looking at predicting ground time to observe and airports effect on a flight.
This would also benefit for scheduling and mapping out hub banks. (Hub banks refering to 'banking' flights in hub and spoke theory)

\subsection{Method 1}
The first method being applied for this research question is multiple linear regression.

\subsection{Method 2}
The second method being applied for this research question is a random forest decision tree (regression).


\section{Research Question 3}
Does airport congestion make the air travel system more prone to delays?

This final question puts a focus on the operations around a flight, specifically late arrival aircraft. If a late aircraft can be predicted an aircraft can be swapped out,
or have a flight schedule relaying more variance in scheduled time. There will be two outputs, there is a late aircraft, or not.

\subsection{Method 1}
The first method being applied for this research question is logistic regression.

\subsection{Method 2}
The second method being applied for this research question is a random forest decision tree.

\section{Data}
For the models in this research question we will provide:

Airline

Origin

Departure Block

Departure Delay

Destination

Arrival Block

Arrival Delay (Time in minutes)

Arrival Delay (True or False)

Distance 

Carrier Delay

Late Aircraft Delay

Air System Delay

Security Delay

Weather Delay

\end{document} 