\documentclass[a4paper,11pt]{article}

%You will not generally need all of these, but...
\usepackage{graphicx} %Allows us to use figures
% \usepackage{fullpage} %Sets all margins to 1.5 cm
% %\usepackage[top=1in, bottom=1in, left=1in, right=1in]{geometry} %Could use this for custom margins
% \usepackage{amsfonts} %More fonts
% \usepackage{amsmath} %More math
\usepackage{subfigure} %Allows subfigures, if we like
% \usepackage{setspace}
% \usepackage{epigraph}
% \usepackage{colortbl} %Allows the fancy color table below
% \usepackage{xcolor}
% \usepackage{array,booktabs}

%Always begin the document with...
\begin{document}

%Title and abstract...
\title{ORD Flight Delays EDA}
\author{Aden Ramirez}
\maketitle

% \begin{abstract}
% \end{abstract}

\section{Important Terms}
This contains some terms and information that will be used throughout the EDA that may not be common knowledge to help you understand. 

\textbf{Delay} - A flight delay is defined by the Federal Aviation Administration (FAA) as a flight arriving 15 minutes or longer after scheduled arrivale time

\textbf{FAA} - Federal Aviation Administration is (FAA) is the governing body of all aviation in the U.S from aircraft certification to air traffic control

\textbf{DOT} - Department of Transportation (DOT) is a governing body that regulates all transportation, in this case they offer many unique identifiers 
and are the parent body to the Bureau of Transportation Statistics (BTS) that the data is sourced from.

\textbf{IATA} - International Air Transport Association (IATA) is an organization that makes a standard for transport type aircraft and in our case create the airport codes referenced
throughout this research. 

\textbf{ICAO} - International Civil Aviation Organization (ICAO) is a United Nations body that has standards for worldwide air travel. They are less important in this research as we look at domestic travel,
they do play in role in how the U.S forms their regulations.

\section{The Data} \label{data}
This data was collected from the Bureau of Transportation Statistics (BLS) as part of the online Airline On-Time Arrival Performance Data. 
I selected one year of data to begin, choosing the most recent as of Janurary 2023. The data range spans all U.S. Carriers flying domestically over the course of November 2021 - November 2022.
The data has many fields totaling to 120 columns. BLS provides an excellent readme that explains each data field, though I will highlight some very important ones to my research here.

\emph{Note: \_ is not shown in these data types as they appear in the .CSV}

\subsection{Basics}
There are alot of columns that are basic information that would be expected of most sets, there is alot of options in this set with plenty of formating opertunity.

\textbf{Year} - Year (4 digit)

\textbf{Quarter} - Quarter of the year (1-4)

\textbf{Month} - Month

\textbf{DayofMonth} - Day date of the month

\textbf{DayofWeek} - Day in words, such as 'Monday'

\textbf{FlightDate} - Flight Date Aggregation (yyyymmdd)

\subsection{Indentifiers}
There are many indentifiers included that are not entirely useful for my use as a DOT Marketing ID, but worth having should there be NA data columns that can be cross referenced.
There is also multiple ways to indentify \emph{anything} in aviation, so there are international Air Transport Association(IATA) and US identifiers in this set.
The most important for my analysis are:

\textbf{Marketing Airline Network} - These are common codes for airlines someone would see on their ticket, such as UA for United Airlines 

\textbf{Tail Number} - Aircraft Tail Number, can uniquely identify a register aircraft

\textbf{Flight Number Operating Airline} - Flight number as you would see on a ticket for example: UA1234

\subsection{Flight Information}

\subsubsection{Route Information}
All the information pertaining to the route of the flight:

\textbf{Origin} - Origin Airport in IATA standard (3 Letter) Example: ORD or PHX

\textbf{OriginCityName} - Origin Airport City (There are special cases such as CLT)

\textbf{OriginState} - Origin State Code

\textbf{OriginStateFips} - Origin State FIPS code, this will allow for adding state geometry in maps

\textbf{OriginStateName} - Origin State Name string

\textbf{Dest} - Destination Airport in IATA standard

\textbf{DestCityName} - Destination Airport City

\textbf{DestState} - Destination State Code

\textbf{DestStateFips} - Destination State FIPS code, this will allow for adding state geometry in maps

\textbf{DestStateName} - Destination State Name string

\subsubsection{Delay Information}
All the information pertaining to the timing of the flight, and delay if applicable. There is much more, but this is a good focus for the analysis and EDA.
Extensive diversion information is not included, as it is noted, but not the focus of the research.

\textbf{CRSDepTime} - CRS Departure Time (local time: hhmm)

\textbf{DepTime} - Actual Departure Time (local time: hhmm)

\textbf{DepDelay} - Difference in minutes between scheduled and actual departure time

\textbf{DepDelayMinutes} - Difference in minutes between scheduled and actual departure time

\textbf{DepDel15} - Departure Delay Indicator if the flight is at least 15 minutes delayed

\textbf{DepartureDelayGroups} - Departure Delay intervals, every (15 minutes from <-15 to >180)

\textbf{DepTimeBlk} - CRS Departure Time Block, Hourly Intervals

\textbf{TaxiOut} - Taxi Out Time, in Minutes

\textbf{WheelsOff} - Wheels Off Time (local time: hhmm)

\textbf{WheelsOn} - Wheels On Time (local time: hhmm)

\textbf{TaxiIn} - Taxi In Time, in Minutes

\textbf{CRSArrTime} - CRS Arrival Time (local time: hhmm)

\textbf{ArrTime} - Actual Arrival Time (local time: hhmm)

\textbf{ArrDelay} - Difference in minutes between scheduled and actual arrival time. Early arrivals show negative numbers.

\textbf{ArrDelayMinutes} - Difference in minutes between scheduled and actual arrival time. Early arrivals set to 0.

\textbf{ArrDel15} - Arrival Delay Indicator, 15 Minutes or More (1=Yes)

\textbf{ArrivalDelayGroups} - Arrival Delay intervals, every (15 minutes from $\leq$ 15 to $>$ 180 minutes )

\textbf{ArrTimeBlk} - CRS Arrival Time Block, Hourly Intervals

\textbf{Cancelled} - Cancelled Flight Indicator (1=Yes)

\textbf{CancellationCode} - Specifies The Reason For Cancellation

\textbf{Diverted} - Diverted Flight Indicator (1=Yes)

\textbf{CRSElapsedTime} - CRS Elapsed Time of Flight, in Minutes

\textbf{ActualElapsedTime} - Elapsed Time of Flight, in Minutes

\textbf{AirTime} - Flight Time, in Minutes

\textbf{Flights} - Number of Flights

\textbf{Distance} - Distance between airports (miles)

\textbf{DistanceGroup} - Distance Intervals, every 250 Miles, for Flight Segment

\textbf{CarrierDelay} - Carrier Delay, in Minutes

\textbf{WeatherDelay} - Weather Delay, in Minutes

\textbf{NASDelay} - National Air System Delay, in Minutes

\textbf{SecurityDelay} - Security Delay, in Minutes

\section{Data Cleaning}

\subsection{Aggregation}
Collecting the initial one year (11/21-11/22) data from BTS yields twelve seperate .zip files containing a .csv file and .html documentation readme.
The first step to using this data is to concatanate all this information together into one usable format.
I accomplish this with a Python script using the \emph{Pandas} library. This script pulls in all .csv files into a list of Pandas Dataframes.
Once these are Pandas Dataframes it is quite simple with all variables matching, they get concatanated. 
Now with a single monolithic Dataframe, it is written to a new .csv file.

\subsection{Subset Files} \label{datasubset}
This research is focusing on a single airport, Chicago O'Hare International Airport. The data sourced is for all U.S Airline Carriers that fly domestically.
The first step in cleaning this data is to properly subset the information into a smaller, less bloated file to reference. My approach will produce to files as output.

First, one main file will use the \emph{Pandas} library in Python to pull in our single monolithic csv file into a Dataframe. From here it is quite simple to mask only the data we want.
I filter by either the \emph{Origin} or \emph{Dest} to be the airport of interest ORD.
Once this mask is applied, the values are verfied to be an expected value, and once again exported using the next masked dataframe to a seperate .csv file. 

Second, a second file that contains only variables being heavily used, to help with operation speed. All the variables listed in section \ref{data} will be maintained.
The process will follow the same process as section \ref{datasubset}. The output will be a much more compact file to operate on, to aid speed for the EDA and future model development.

\section{Exploratory Data Analysis}

\begin{figure}
    \centering
    \includegraphics*[scale=.5]{../../img/ord_delay_split_pie.png}
    \caption[Delays by Type]{ORD Delays by Type}
    \label{fig:delaybytype}
\end{figure}

\begin{figure}
    \centering
    \includegraphics*[scale=.40]{../../img/ord_delay_t_month.png}
    \caption[]{}
    \label{fig:delaybymonth}
\end{figure}

\begin{figure}
    \centering
    \includegraphics*[scale=.40]{../../img/delays_by_airline.png}
    \caption[]{}
    \label{fig:delaybyairline}
\end{figure}

\begin{figure}
    \centering
    \includegraphics*[scale=.40]{../../img/flights_by_airline.png}
    \caption[]{}
    \label{fig:flightsbyairline}
\end{figure}

\begin{figure}
    \centering
    \includegraphics*[scale=.40]{../../img/dist_by_delay.png}
    \caption[]{}
    \label{fig:lm_disbydelay}
\end{figure}

\begin{figure}
    \centering
    \includegraphics*[scale=.40]{../../img/airtime_by_delay.png}
    \caption[]{}
    \label{fig:lm_airtimebydelay}
\end{figure}

\end{document} 