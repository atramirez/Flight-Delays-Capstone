\documentclass[a4paper,12pt]{article}

\begin{document}

\title{ORD Flight Delays Ethical Considerations}
\author{Aden Ramirez}
\maketitle

% \pagebreak

% Ideas for some things to condsider
% Overbooking
% Longer cancellation
% Cancelling or moving flights too much (over adjusting)

There are some ethical considerations for the tooling and analysis developed in this project looking closer at O'Hare International Airport flight delays.
Predicting flight delays is hugely beneficial to the industry, allowing for carriers to plan ahead and work towards eliminating preventable delays.
With this information though there are so problems that could arise such as delaying a flight that needs to be cancelled, prolonging the issues for the customer,
or over adjusting for predicted delays by moving the flight time dramatically to attempt to avoid a delay.

Airlines are widely known to have a low margin for profits, cutting costs, and maximizing money from flights. Carriers want to avoid any extra costs possible especially 
having to accomadate a passenger who's flight has been cancelled. Predicting flight delays, and getting the precision down to a block, even if the block is large could
allow for an airline to hold out cancelling a flight in the name of cost savings, since they believe they can see it would be a long delay, or even just extend the time, being
more inconvenient for the passengers. This would not necessarily be industry breaking, but would leave an oppertunity for airlines to skirt costs and create situations
that would not be in the favor of the customer.

Another consideration that is closely related, predicting flight delays, especially when looking at block departure or arrival times historically airlines would likely
adjust their schedules each year to account for this. This is common, and isn't bad in itself, flights are large airports often have more padding with a higher average taxi
time. This does leave the ability to "over-adjust" when seeing a flight it likely to be delayed where they could add a later arrival time not only to pad from a long taxi, but 
an anticipated delay. This over adjustment could lead to longer flight (total not in air) times that oculd have an effect throughout the system, but not actually having 
a delayed flight since iwth new pad, it is technically on time or even early if the adjustment becomes too extreme. 

Carriers reputation is important to their business with a low margin, highly competitive industry, they will take every chance to increase their reputation or cut cost.
With this analysis and implmentation this information could be used to assist and airline with that, and of course could effect the future of airline schedules where total flight
time is increased more than it should be in order to padding time to avoid delays. The information must be use responsibily and not adding unnecessary time to flights, that
would cause these problems.

\end{document} 